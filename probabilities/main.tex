\documentclass{article}

\usepackage{biblatex}
\addbibresource{main.bib}

\begin{document}

\section{Introduction}
Routers can be bad. We want to audit them. Yada yada.

%%%%%% Methods %%%%%%
\section{Hash Digest Approach}
\subsection{Method}
As the ISP receives packets, it maintains a streaming cryptographic multiset
hash~\cite{clarke2003incremental}.

Let $n$ be the number of packets in the router digest, $r$ be the number
received by the ISP.

During an audit, we first check if $n - r$ is in a predetermined constant
range. Then we consider all $\binom{n}{n - r}$ subsets of the router log $L$,
and hash each one. The audit passes if at least one of those subsets hashes to
the ISP's multiset hash value.

\subsection{Analysis}
We assume the router controls its log $L$, as well as the set $I$ of packets
the ISP receives. We want to know how hard it is to construct an $I, L$ such
that $I \not\subset L$ but there is some $F \subset L$ such that $I$ and $F$
collide under the multiset hash.

\cite{clarke2003incremental} can be summarized as follows:
\begin{itemize}
    \item `A multiset hash function is multiset-collision resistant if it is
        computationally infeasible to find multisets $S, M$ such that the
        cardinalities of $S$ and $M$ are of polynomial size in $m$, $S \neq M$,
        and they are equivalent under the multiset hash.'
    \item Specifically, the game is we pick a key for the hash, then the ppt
        (in $m$) adversary tries to generate $(S, M)$ that collide. The
        probability of this should be less than $m^{-c}$, where $m$ is the
        bitlength of the hash output and $c$ is a constant.
\end{itemize}

So let's see if we can get a reduction. We want to show that if the router
could break our scheme, they could win the multiset collision game. So, assume
the router has constructed an $I, L$ such that $I \not\subset L$ but there
exists an $F \subset L$ such that $I$ and $F$ collide under the multiset hash.
That seems like a collision, but the problem is there are $2^L$ possible
subsets, so actually \emph{finding} $F$ may be infeasible.

However, if we assume the number of dropped packets is bounded by a constant,
then there are only polynomially many such subsets and they can be enumerated
(and checked) in polynomial time. So that would give us an algorithm for
collision, so our reduction should work.

I think here the polynomial is in $m$, which is (essentially) the size of our
hash.

More generally, even with the heuristics below, I think what we can claim is:
if the auditing process runs in polynomial time (wrt the size of the hash),
then it's secure. Because otherwise the router could do the audit itself, find
the collision, and win the game in ppt.

%%%%%% Power Series %%%%%%
\section{Power Series Heuristic}
\subsection{Method}
For a threshold $t$ of dropped packets, the ISP keeps track of power sums $p_i$
up to $t$ of the packets it receives.

Upon audit, the power sums $q_i$ from the log are computed as well. Then we
compute $q_i - p_i$ to get the power sums of the dropped packets. This system
is solved to identify the dropped packets $D$. Then we compute the hash-sum of
the digest hash with the hash of the dropped packets to get the hash of the
multi-set union of the two, and compare that to the hash of the log.

\subsection{Analysis}
If the router is honest, the final comparison will pass.

Based on the analysis in the previous section, all we need to do is show that
the auditing process runs in time polynomial to $m$, the bitlength of the hash.

Suppose we want to support a log of size at most $l$, at most $t$ dropped
packets, and we choose a message length of size $l + t$. Then all the steps
(computing power sums from the log, subtracting the digest from those, solving
the system, computing the hash of the difference, adding to the original hash,
and comparing) should all take poly time.

%%%%%% IBLT %%%%%%
\section{IBLT Heuristic}
\subsection{Method}

\subsection{Analysis}

\printbibliography

\end{document}
